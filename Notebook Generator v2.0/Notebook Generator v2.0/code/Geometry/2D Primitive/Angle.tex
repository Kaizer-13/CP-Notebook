A class for ordering angles (as represented by int points and a number of rotations around the origin). Useful for rotational sweeping.
Sometimes also represents points or vectors.
\begin{minted}{C++}
/* Usage:
 *  vector<Angle> v = {w[0], w[0].t360() ...}; // sorted
 *  int j = 0; rep(i,0,n) { while (v[j] < v[i].t180()) ++j; }
 *  // sweeps j such that (j-i) represents the number of positively oriented triangles with vertices at 0 and i
*/
struct Angle {
  int x, y;
  int t;
  Angle(int x, int y, int t=0) : x(x), y(y), t(t) {}
  Angle operator-(Angle b) const { return {x-b.x, y-b.y, t}; }
  int half() const {
    assert(x || y);
    return y < 0 || (y == 0 && x < 0);
  }
  Angle t90() const { return {-y, x, t + (half() && x >= 0)}; }
  Angle t180() const { return {-x, -y, t + half()}; }
  Angle t360() const { return {x, y, t + 1}; }
};
bool operator<(Angle a, Angle b) {
  // add a.dist2() and b.dist2() to also compare distances
  return make_tuple(a.t, a.half(), a.y * (ll)b.x) < make_tuple(b.t, b.half(), a.x * (ll)b.y);
}

// Given two points, this calculates the smallest angle between
// them, i.e., the angle that covers the defined line segment.
pair<Angle, Angle> segmentAngles(Angle a, Angle b) {
  if (b < a) swap(a, b);
  return (b < a.t180() ? make_pair(a, b) : make_pair(b, a.t360()));
}
Angle operator+(Angle a, Angle b) { // point a + vector b
  Angle r(a.x + b.x, a.y + b.y, a.t);
  if (a.t180() < r) r.t--;
  return r.t180() < a ? r.t360() : r;
}
Angle angleDiff(Angle a, Angle b) { // angle b - angle a
  int tu = b.t - a.t; a.t = b.t;
  return {a.x*b.x + a.y*b.y, a.x*b.y - a.y*b.x, tu - (b < a)};
}
\end{minted}
