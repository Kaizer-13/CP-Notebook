Line-convex polygon intersection. The polygon must be ccw and have no collinear points.
lineHull(line, poly) returns a pair describing the intersection of a line with the polygon:
\begin{itemize*}
\item $(-1, -1)$ if no collision,
\item $(i, -1)$ if touching the corner $i$,
\item $(i, i)$ if along side $(i, i+1)$,
\item $(i, j)$ if crossing sides $(i, i+1)$ and $(j, j+1)$.
\end{itemize*}
In the last case, if a corner $i$ is crossed, this is treated as happening on side $(i, i+1)$.
The points are returned in the same order as the line hits the polygon.
\texttt{extrVertex} returns the point of a hull with the max projection onto a line.\\
Time: $O(\log n)$
\begin{minted}[breaklines = true,
    breakanywhere = true,
    frame=lines,    
    fontfamily=tt,
    linenos=false,
    numberblanklines=true,
    numbersep=2pt,
    gobble=0,
    framerule=1pt,
    framesep=1mm,
    funcnamehighlighting=true,
    tabsize=1,
    obeytabs=false,
    mathescape=false
    samepage=false, %with this setting you can force the list to appear on the same page
    showspaces=false,
    showtabs =false,
    texcl=false]{C++}
#include "Point.h"

#define cmp(i,j) sgn(dir.perp().cross(poly[(i)%n]-poly[(j)%n]))
#define extr(i) cmp(i + 1, i) >= 0 && cmp(i, i - 1 + n) < 0
template <class P> int extrVertex(vector<P>& poly, P dir) {
  int n = sz(poly), lo = 0, hi = n;
  if (extr(0)) return 0;
  while (lo + 1 < hi) {
    int m = (lo + hi) / 2;
    if (extr(m)) return m;
    int ls = cmp(lo + 1, lo), ms = cmp(m + 1, m);
    (ls < ms || (ls == ms && ls == cmp(lo, m)) ? hi : lo) = m;
  }
  return lo;
}

#define cmpL(i) sgn(a.cross(poly[i], b))
template <class P>
array<int, 2> lineHull(P a, P b, vector<P>& poly) {
  int endA = extrVertex(poly, (a - b).perp());
  int endB = extrVertex(poly, (b - a).perp());
  if (cmpL(endA) < 0 || cmpL(endB) > 0)
    return {-1, -1};
  array<int, 2> res;
  rep(i,0,2) {
    int lo = endB, hi = endA, n = sz(poly);
    while ((lo + 1) % n != hi) {
      int m = ((lo + hi + (lo < hi ? 0 : n)) / 2) % n;
      (cmpL(m) == cmpL(endB) ? lo : hi) = m;
    }
    res[i] = (lo + !cmpL(hi)) % n;
    swap(endA, endB);
  }
  if (res[0] == res[1]) return {res[0], -1};
  if (!cmpL(res[0]) && !cmpL(res[1]))
    switch ((res[0] - res[1] + sz(poly) + 1) % sz(poly)) {
      case 0: return {res[0], res[0]};
      case 2: return {res[1], res[1]};
    }
  return res;
}
\end{minted}
