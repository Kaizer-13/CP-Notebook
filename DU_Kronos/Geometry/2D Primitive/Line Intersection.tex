\begin{minipage}{75mm}
If a unique intersection point of the lines going through s1,e1 and s2,e2 exists \{1, point\} is returned.
If no intersection point exists \{0, (0,0)\} is returned and if infinitely many exists \{-1, (0,0)\} is returned.
The wrong position will be returned if P is Point<ll> and the intersection point does not have integer coordinates.
Products of three coordinates are used in intermediate steps so watch out for overflow if using int or ll.
\end{minipage}
\begin{minipage}{15mm}
\includegraphics[width=\textwidth]{"../code/Geometry/2D Primitive/lineIntersection"}
\end{minipage}
\begin{minted}{C++}
/* Usage:
 *  auto res = lineInter(s1,e1,s2,e2);
 *  if (res.first == 1)
 *    cout << "intersection point at " << res.second << endl;
 */
#pragma once

#include "Point.h"

template<class P>
pair<int, P> lineInter(P s1, P e1, P s2, P e2) {
  auto d = (e1 - s1).cross(e2 - s2);
  if (d == 0) // if parallel
    return {-(s1.cross(e1, s2) == 0), P(0, 0)};
  auto p = s2.cross(e1, e2), q = s2.cross(e2, s1);
  return {1, (s1 * p + e1 * q) / d};
}
\end{minted}
