Class to handle points in 3D space.T can be e.g. double or long long.
\begin{minted}[breaklines = true,
    breakanywhere = true,
    frame=lines,    
    fontfamily=tt,
    linenos=false,
    numberblanklines=true,
    numbersep=2pt,
    gobble=0,
    framerule=1pt,
    framesep=1mm,
    funcnamehighlighting=true,
    tabsize=1,
    obeytabs=false,
    mathescape=false
    samepage=false, %with this setting you can force the list to appear on the same page
    showspaces=false,
    showtabs =false,
    texcl=false]{C++}
template<class T> struct Point3D {
  typedef Point3D P;
  typedef const P& R;
  T x, y, z;
  explicit Point3D(T x=0, T y=0, T z=0) : x(x), y(y), z(z) {}
  bool operator<(R p) const {
    return tie(x, y, z) < tie(p.x, p.y, p.z); }
  bool operator==(R p) const {
    return tie(x, y, z) == tie(p.x, p.y, p.z); }
  P operator+(R p) const { return P(x+p.x, y+p.y, z+p.z); }
  P operator-(R p) const { return P(x-p.x, y-p.y, z-p.z); }
  P operator*(T d) const { return P(x*d, y*d, z*d); }
  P operator/(T d) const { return P(x/d, y/d, z/d); }
  T dot(R p) const { return x*p.x + y*p.y + z*p.z; }
  P cross(R p) const {
    return P(y*p.z - z*p.y, z*p.x - x*p.z, x*p.y - y*p.x);
  }
  T dist2() const { return x*x + y*y + z*z; }
  double dist() const { return sqrt((double)dist2()); }
  //Azimuthal angle (longitude) to x-axis in interval [-pi, pi]
  double phi() const { return atan2(y, x); } 
  //Zenith angle (latitude) to the z-axis in interval [0, pi]
  double theta() const { return atan2(sqrt(x*x+y*y),z); }
  P unit() const { return *this/(T)dist(); } //makes dist()=1
  //returns unit vector normal to *this and p
  P normal(P p) const { return cross(p).unit(); }
  //returns point rotated 'angle' radians ccw around axis
  P rotate(double angle, P axis) const {
    double s = sin(angle), c = cos(angle); P u = axis.unit();
    return u*dot(u)*(1-c) + (*this)*c - cross(u)*s;
  }
};
\end{minted}
