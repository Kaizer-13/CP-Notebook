Finds the external tangents of two circles, or internal if r2 is negated.
Can return 0, 1, or 2 tangents -- 0 if one circle contains the other (or overlaps it, in the internal case, or if the circles are the same);
1 if the circles are tangent to each other (in which case .first = .second and the tangent line is perpendicular to the line between the centers).
.first and .second give the tangency points at circle 1 and 2 respectively.
To find the tangents of a circle with a point set r2 to 0.
\begin{minted}[breaklines = true,
    breakanywhere = true,
    frame=lines,    
    fontfamily=tt,
    linenos=false,
    numberblanklines=true,
    numbersep=2pt,
    gobble=0,
    framerule=1pt,
    framesep=1mm,
    funcnamehighlighting=true,
    tabsize=1,
    obeytabs=false,
    mathescape=false
    samepage=false, %with this setting you can force the list to appear on the same page
    showspaces=false,
    showtabs =false,
    texcl=false]{C++}
#include "Point.h"

template<class P>
vector<pair<P, P>> tangents(P c1, double r1, P c2, double r2) {
  P d = c2 - c1;
  double dr = r1 - r2, d2 = d.dist2(), h2 = d2 - dr * dr;
  if (d2 == 0 || h2 < 0)  return {};
  vector<pair<P, P>> out;
  for (double sign : {-1, 1}) {
    P v = (d * dr + d.perp() * sqrt(h2) * sign) / d2;
    out.push_back({c1 + v * r1, c2 + v * r2});
  }
  if (h2 == 0) out.pop_back();
  return out;
}
\end{minted}
